%%%%%%%%%%%%%%%%%%%%%%%%%%%%%%%%%%%%%%%%%%%%%%%%%%%%%%%%%%%%%%%%%%%%%%%%%%%%%%%
% A clean template for an academic CV
%
% Uses tabularx to create two column entries (date and job/edu/citation).
% Defines commands to make adding entries simpler.
%
%%%%%%%%%%%%%%%%%%%%%%%%%%%%%%%%%%%%%%%%%%%%%%%%%%%%%%%%%%%%%%%%%%%%%%%%%%%%%%%

\documentclass[10pt, a4paper]{article}

% Full Unicode support for non-ASCII characters
\usepackage[utf8]{inputenc}

% Useful aliases
\newcommand{\JHU}{Johns Hopkins University}
\newcommand{\PC}{Presbyterian College}
\newcommand{\DOC}{Department of Chemistry}
\newcommand{\DOP}{The William H. Miller III Department of Physics and Astronomy, Institute for Quantum Matter}
\newcommand{\ORNL}{Oak Ridge National Laboratory}
\newcommand{\CSD}{Chemical Sciences Division}

% Identifying information
\newcommand{\Title}{Curriculum Vit\ae}
\newcommand{\FirstName}{Lucas}
\newcommand{\LastName}{Pressley}
\newcommand{\Initials}{LA}
\newcommand{\MyName}{Dr. \Lucas\ \Pressley}
\newcommand{\Me}{\textbf{\Pressley, \Initials}}  % For citations
\newcommand{\Email}{lucaspressley@gmail.com}
\newcommand{\ORCID}{0000-0002-0765-3333}

% Names for citing coauthors
\newcommand{\Tanya}{Berry, T}
\newcommand{\Phelan}{Phelan, WA}
\newcommand{\Thao}{Tran, TT}
\newcommand{\McQueen}{McQueen, TM}
\newcommand{\Mekhola}{Sinha, M}
\newcommand{\Hector}{Vivanco, HK}
\newcommand{\Cheng}{Wan, Cheng}
\newcommand{\Max}{Siegler, MA}
\newcommand{\Veronica}{Stewart, VJ}
\newcommand{\Lisa}{Pogue, EA}
\newcommand{\Ziqian}{Wang, Z}
\newcommand{\Johnson}{Johnson, I}
\newcommand{\Mingwei}{Chen, M}
\newcommand{\Juan}{Chamorro, JR}
\newcommand{\Prashant}{Chauhan, P}
\newcommand{\Cannon}{Sun, C}
\newcommand{\Nichol}{Varnava, N}
\newcommand{\Winiarski}{Winiarski, MJ}
\newcommand{\Nick}{Ng, Nicholas}
\newcommand{\YLi}{Li, Y}
\newcommand{\Chris}{Pasco, CM}
\newcommand{\Vanderbilt}{Vanderbilt, D}
\newcommand{\Armitage}{Armitage, NP}
\newcommand{\CJ}{Wright, CJ}
\newcommand{\Eric}{Dooryhee, E}
\newcommand{\Khalifah}{Khalifah, PG}
\newcommand{\Billinge}{Billinge, SJL}
\newcommand{\Ali}{Ghasemi, Alireza}
\newcommand{\Sieun}{Chae, S}
\newcommand{\Hanjong}{Paik, H}
\newcommand{\Gim}{Gim, J}
\newcommand{\Goodge}{Goodge, BH}
\newcommand{\Werder}{Werder, D}
\newcommand{\Lena}{Kourkoutis, LF}
\newcommand{\Hovden}{Hovden, R}
\newcommand{\Kioupakis}{Kioupakis, E}
\newcommand{\Heron}{Heron, JT}
\newcommand{\Sujit}{Das, S}
\newcommand{\Ramesh}{Ramamoorthy, R}
\newcommand{\Mizzi}{Mizzi, C}
\newcommand{\Harrison}{Harrison, Neil}
\newcommand{\Rosa}{Rosa, P}
\newcommand{\Chan}{Chan, Mun}
\newcommand{\YiLuo}{Luo, Yi}
\newcommand{\Jonathan}{Gaudet, J}
\newcommand{\Krajewska}{Krajewska, A}
\newcommand{\Helen}{Walker, H}
\newcommand{\Predrag}{Nikolic, P}
\newcommand{\Collin}{Broholm, C}
\newcommand{\KMoerman}{Moerman, KM}
\newcommand{\KNiemeyer}{Niemeyer, KE}
\newcommand{\JPoulson}{Poulson, JL}
\newcommand{\PPrins}{Prins, P}
\newcommand{\KRam}{Ram, K}
\newcommand{\ARokem}{Rokem, A}
\newcommand{\Arfon}{Smith, AM}
\newcommand{\GThiruvathukal}{Thiruvathukal, GK}
\newcommand{\KThyng}{Thyng, KM}
\newcommand{\BWilson}{Wilson, BE}
\newcommand{\Yehudi}{Yehudi, Y}
\newcommand{\Remi}{Rampin, R}
\newcommand{\Hugo}{van Kemenade, H}
\newcommand{\MattTurk}{Turk, M}
\newcommand{\Shapero}{Shapero, D}
\newcommand{\Anderson}{Banihirwe, A}
\newcommand{\Leeman}{Leeman, J}
\newcommand{\JEbbing}{Ebbing, J}
\newcommand{\AGuy}{Guy, A}
\newcommand{\JFarquharson}{Farquharson, J}
\newcommand{\AKushnir}{Kushnir, A}
\newcommand{\FWadsworth}{Wadsworth, F}
\newcommand{\LPerozzi}{Perozzi, L}
\newcommand{\MWieczorek}{Wieczorek, MA}
\newcommand{\LLi}{Li, L}
\newcommand{\Ricardo}{Trindade, RIF}


% Template configuration
%%%%%%%%%%%%%%%%%%%%%%%%%%%%%%%%%%%%%%%%%%%%%%%%%%%%%%%%%%%%%%%%%%%%%%%%%%%%%%%

% Disable hyphenation
\usepackage[none]{hyphenat}

% Control the font size
\usepackage{anyfontsize}

% Icon fonts (requires using xelatex or luatex)
\usepackage[fixed]{fontawesome5}
\usepackage{academicons}

% Template variables for styling
\newcommand{\TablePad}{\vspace{-0.4cm}}
\newcommand{\SoftwareTitle}[1]{{\bfseries #1}}
\newcommand{\TableTitle}[1]{{\fontsize{12pt}{0}\selectfont \itshape #1}}

% For fancy and multipage tables
\usepackage{tabularx}
\usepackage{ltablex}

% Define a new environment to place all CV entries in a 2-column table.
% Left column are the dates, right column the entries.
\usepackage{environ}
\NewEnviron{EntriesTable}{
\TablePad
\begin{tabularx}{\textwidth}{@{}p{0.10\textwidth}@{\hspace{0.02\textwidth}}p{0.88\textwidth}@{}}
  \BODY
\end{tabularx}
}
\NewEnviron{EntriesTableExtra}{
\TablePad
\begin{tabularx}{\textwidth}{@{}p{0.10\textwidth}@{\hspace{0.02\textwidth}}p{0.79\textwidth}@{\hspace{0.02\textwidth}}>{\raggedright\arraybackslash}p{0.07\textwidth}}
  \BODY
\end{tabularx}
}

% Macros to add links and mark publications
\newcommand{\DOI}[1]{doi:\href{https://doi.org/#1}{#1}}
\newcommand{\DOILink}[1]{\href{https://doi.org/#1}{doi.org/#1}}
\newcommand{\Website}[1]{\href{https://#1}{#1}}
\newcommand{\Preprint}[1]{\href{https://doi.org/#1}{\faFilePdf}}
\newcommand{\Youtube}[1]{\href{https://www.youtube.com/watch?v=#1}{\faYoutube}}
\newcommand{\GitHub}[1]{\href{https://github.com/#1}{\faGithub}}
\newcommand{\Data}[1]{\href{https://doi.org/#1}{\faChartLine}}
\newcommand{\Slides}[1]{\href{https://#1}{\faTv}}
\newcommand{\SlidesDOI}[1]{\href{https://doi.org/#1}{\faTv}}
\newcommand{\PosterDOI}[1]{\href{https://doi.org/#1}{\faImage}}
\newcommand{\OA}{\thinspace\aiOpenAccess\enspace}

% Macros to set the year and duration on the left column
\newcommand{\Duration}[2]{\fontsize{9pt}{0}\selectfont #1 -- #2}
\newcommand{\Year}[1]{\fontsize{9pt}{0}\selectfont #1}
\newcommand{\Ongoing}{on}
\newcommand{\Future}{future}
\newcommand{\Appointment}[4]{\textbf{#1} \newline #2 \newline #3 \newline #4}

% Define command to insert month name and year as date
\usepackage{datetime}
\newdateformat{monthyear}{\monthname[\THEMONTH], \THEYEAR}

% Set the page margins
\usepackage[a4paper,margin=1.5cm,includehead,headsep=5mm]{geometry}

% To get the total page numbers (\pageref{LastPage})
\usepackage{lastpage}

% No indentation
\setlength\parindent{0cm}

% Increase the line spacing
\renewcommand{\baselinestretch}{1.2}
% and the spacing between rows in tables
\renewcommand{\arraystretch}{1.5}

% Remove space between items in itemize and enumerate
\usepackage{enumitem}
\setlist{nosep}

% Use custom colors
\usepackage[usenames,dvipsnames]{xcolor}

% Set fonts (requires compilation with xelatex)
\usepackage{fontspec}
\setmainfont[%
  Path = fonts/notoserif/,
  UprightFont = NotoSerif-Regular,
  BoldFont = NotoSerif-Bold,
  ItalicFont = NotoSerif-Italic,
  Extension = .ttf
]{NotoSerif}



% Set the spacing for sections
\usepackage{titlesec}
\titleformat{\section}
  {\normalfont\Large\mdseries} % format
  {} % label
  {0pt} % separation (left separation for hang)
  {} % text before title
  [\titlerule] % text after title
\titleformat{\subsection}
  {\normalfont\large\mdseries} % format
  {} % label
  {0pt} % separation (left separation for hang)
  {} % text before title

% Disable number of sections. Use this instead of "section*" so that the sections still
% appear as PDF bookmarks. Otherwise, would have to add the table of contents entries
% manually.
\makeatletter
\renewcommand{\@seccntformat}[1]{}
\makeatother

% Set fancy headers
\usepackage{fancyhdr}
\pagestyle{fancy}
\fancyhf{}
\lhead{\fontsize{9pt}{10pt}\selectfont
  \monthyear\today
}
\chead{
  \fontsize{9pt}{10pt}\selectfont
  \MyName
  \hspace{0.2cm} -- \hspace{0.2cm}
  \Title
}
\rhead{\fontsize{9pt}{10pt}\selectfont \thepage{} of \pageref*{LastPage}}
\renewcommand{\headrulewidth}{0pt}

% Metadata for the PDF output and control of hyperlinks
\usepackage[colorlinks=true]{hyperref}
\hypersetup{
  pdftitle={\MyName\ - \Title},
  pdfauthor={\MyName},
  linkcolor=blue,
  citecolor=blue,
  filecolor=black,
  urlcolor=MidnightBlue
}
%%%%%%%%%%%%%%%%%%%%%%%%%%%%%%%%%%%%%%%%%%%%%%%%%%%%%%%%%%%%%%%%%%%%%%%%%%%%%%%


\begin{document}

% No header for the first page
\thispagestyle{empty}

%%%%%%%%%%%%%%%%%%%%%%%%%%%%%%%%%%%%%%%%%%%%%%%%%%%%%%%%%%%%%%%%%%%%%%%%%%%%%%%
\begin{minipage}[t]{0.7\textwidth}
{\fontsize{22pt}{0}\selectfont\MyName}
\end{minipage}
\begin{minipage}[t]{0.3\textwidth}
  \begin{flushright}
    Last updated: \monthyear\today
  \end{flushright}
\end{minipage}
\\[-0.1cm]
\rule{\textwidth}{2pt}
\\[0.1cm]
\begin{minipage}[t]{0.7\textwidth}
    ORCID: \href{https://orcid.org/\ORCID}{\ORCID}
    \\
    Email: \href{mailto:\Email}{\Email}
    \\
    Research group: \Website{\LabWebsite}
    \\
    Website: \Website{\PersonalWebsite}
\end{minipage}
\begin{minipage}[t]{0.3\textwidth}
  \begin{flushright}
    \Address
  \end{flushright}
\end{minipage}

%%%%%%%%%%%%%%%%%%%%%%%%%%%%%%%%%%%%%%%%%%%%%%%%%%%%%%%%%%%%%%%%%%%%%%%%%%%%%%%
\section{Professional Appointments}

\begin{EntriesTable}
  \Duration{2022}{\Ongoing}  &
  \Appointment{Postdoctoral Research}{\ORNL}{\CSD}{TN, USA}
  \\
  \Duration{2017}{2022}  &
  \Appointment{Graduate Student Researcher}{\JHU}{\DOC}{MD, USA}
  \\
  \Duration{2016}{2017}  &
  \Appointment{Undergraduate Researcher}{\PC}{\DOC}{SC, USA}
\end{EntriesTable}


%%%%%%%%%%%%%%%%%%%%%%%%%%%%%%%%%%%%%%%%%%%%%%%%%%%%%%%%%%%%%%%%%%%%%%%%%%%%%%%
\section{Education}

\begin{EntriesTable}
  \Duration{2019}{2022}  &
  \textbf{PhD in Chemistry}, Johns Hopkins University, MD, USA
  \\
  \Duration{2017}{2019}  &
  \textbf{MSc in Chemistry}, Johns Hopkins University, MD, USA
  \\
  \Duration{2013}{2017}  &
  \textbf{BSc in Chemistry w/Honors}, Presybterian College, SC, USA
\end{EntriesTable}

\subsection{Reviewer}

\begin{itemize}
  \item Geophysical Journal International
  \item Journal of Geodesy
  \item Pure and Applied Geophysics
  \item Journal of Applied Geophysics
  \item Geophysical Prospecting
  \item Geophysics
  \item Central European Journal of Geosciences
  \item Computers \& Geosciences
  \item Journal of Open Source Software
\end{itemize}

%%%%%%%%%%%%%%%%%%%%%%%%%%%%%%%%%%%%%%%%%%%%%%%%%%%%%%%%%%%%%%%%%%%%%%%%%%%%%%%
\section{Teaching}

\subsection{Teaching Assistantships}

\begin{EntriesTableExtra}
  \Year{2019} &
  Intermediate Organic Chemistry Laboratory
  \newline
  Supervised weekly experiments for class of around 30 students, ran NMR
  samples for students, and graded weekly lab assignments.
  \newline
  \textit{\JHU}
  & ~
  \\
  \Year{2018}  &
  Applied Chemical Equilibrium and Reactivity w/Lab
  \newline
  Supervised weekly experiments for around 40 students, led weekly help
  sessions for students, and graded all written work, including final exams
  \newline
  \textit{\JHU}
  & ~
  \\
  \Duration{2017}{2018}  &
  ENVS386: Introductory Chemistry I and II Lab/Lecture
  \newline
  Led weekly help sessions for students, supervised weekly experiments for around 15 students, graded in person lab skills exams, and graded all
  written work, including final exams.
  \textit{\JHU}
  & ~
  \\
\end{EntriesTableExtra}

\subsection{Workshops \& Short Courses}

\begin{EntriesTableExtra}
\Year{2020}  &
  University of Texas µCT Summer School
\end{EntriesTableExtra}


%%%%%%%%%%%%%%%%%%%%%%%%%%%%%%%%%%%%%%%%%%%%%%%%%%%%%%%%%%%%%%%%%%%%%%%%%%%%%%%
\section{Mentorshipo}

\subsection{PARADIM Research Experience Undergraduate (REU)}

\begin{EntriesTable}
\Duration{2018}{2021}
\end{EntriesTable}

\subsection{Hopkins Extreme Materials Institute (HEMI) Research and Engineering Apprenticeship Program (REAP)}

\begin{EntriesTable}
\Duration{2019}{2020}
\end{EntriesTable}

\subsection{Undergradute Mentor}

\begin{EntriesTable}
\Year{2022}
\end{EntriesTable}

%%%%%%%%%%%%%%%%%%%%%%%%%%%%%%%%%%%%%%%%%%%%%%%%%%%%%%%%%%%%%%%%%%%%%%%%%%%%%%%
\section{Media \& Outreach}

\begin{EntriesTable}
\Duration{2018}{2022} & PARADIM Materials Growth and Design Summer School Group Leader
  \href{https://www.paradim.org/summer_schools}
  \\
\Year{2019}  & Stem Achievement in Baltimore Elementary Schools Fair Volunteer
  \href{http://engineering.jhu.edu/sabes/}
\end{EntriesTable}

%%%%%%%%%%%%%%%%%%%%%%%%%%%%%%%%%%%%%%%%%%%%%%%%%%%%%%%%%%%%%%%%%%%%%%%%%%%%%%%
\section{Publications}

\subsection{Peer-reviewed Papers}

\begin{EntriesTableExtra}
\Year{2022}  &
  \Sieun, \Me, \Hanjong, \Gim, \Werder, \Goodge, \Lena, \Hovden, \McQueen, \Kioupakis, \Heron.
  Germanium dioxide: A new rutile substrate for epitaxial film growth
  \emph{Journal of Vacuum Science & Technology A: Vacuum, Surfaces, and Films}.
  \DOI{10.1116/6.0002011}.
  \\
\Year{2022}  &
  \Me, \Mekhola, \Hector, \Juan, \Sujit, \Ramesh, \McQueen.
  Optimization of PbTiO3 Single Crystals with Flux and Laser Floating Zone Method
  \emph{Crystal Growth & Design}.
  \DOI{10.21105/joss.01943}.
  &
  \OA
  \GitHub{fatiando/pooch}
  \\
\Year{2019}  &
  \Paul, \Joaquim, \Me, \Remko, \Florian, \Walter, \Dongdong.
  The Generic Mapping Tools, Version 6.
  \emph{Geochemistry, Geophysics, Geosystems}.
  \DOI{10.1029/2019GC008515}.
  &
  \OA
  \\
  ~ &
  \Santiago, \Agustina, \Gimenez, \Me.
  Gravitational field calculation in spherical coordinates using variable densities in
  depth.
  \emph{Geophysical Journal International}.
  \DOI{10.1093/gji/ggz277}.
  &
  \GitHub{pinga-lab/tesseroid-variable-density}
  \Preprint{10.31223/osf.io/3548g}
  \\
  ~ &
  \Guangdong, \Bo, \Me, \JLiu, \MKaban, \LChen, \RGuo.
  Efficient 3D large-scale forward-modeling and inversion of gravitational fields in
  spherical coordinates with application to lunar mascons.
  \emph{Journal of Geophysical Research: Solid Earth}.
  \DOI{10.1029/2019jb017691}.
  &
  \Preprint{10.31223/osf.io/dzf9j}
  \\
\Year{2018}  &
  \Me. Verde: Processing and gridding spatial data using Green's functions.
  \emph{Journal of Open Source Software}.
  \DOI{10.21105/joss.00957}.
  &
  \OA
  \GitHub{fatiando/verde}
  \\
\Year{2017}  &
  \Me, \Val.
  Fast non-linear gravity inversion in spherical coordinates with application
  to the South American Moho,
  \emph{Geophysical Journal International},
  \DOI{10.1093/gji/ggw390}.
  &
  \GitHub{pinga-lab/paper-moho-inversion-tesseroids}
  \Preprint{10.31223/osf.io/9ba4m}
  \\
\Year{2016}  &
  \Me, \Val, \Carla.
  Tesseroids: forward modeling gravitational fields in spherical coordinates,
  \emph{Geophysics},
  \DOI{10.1190/geo2015-0204.1}.
  &
  \GitHub{pinga-lab/paper-tesseroids}
  \\
  ~ &
  \Dio, \Me, \Val.
  How two gravity-gradient inversion methods can be used to reveal different
  geologic features of ore deposit - A case study from the Quadrilátero
  Ferrífero (Brazil),
  \emph{Journal of Applied Geophysics},
  \DOI{10.1016/j.jappgeo.2016.04.011}.
  & ~
  \\
\Year{2015}  &
  \Bi, \Dai, \Val, \Me.
  Estimation of the total magnetization direction of approximately spherical
  bodies,
  \emph{Nonlinear Processes in Geophysics},
  \DOI{10.5194/npg-22-215-2015}.
  &
  \OA
  \GitHub{pinga-lab/Total-magnetization-of-spherical-bodies}
  \\
\Year{2014}  &
  \Dio, \Me, \Val.
  Imaging iron ore from the Quadrilátero Ferrífero (Brazil) using geophysical
  inversion and drill hole data,
  \emph{Ore Geology Reviews},
  \DOI{10.1016/j.oregeorev.2014.02.011}.
  & ~
  \\
\Year{2013}  &
  \Figura, \Val, \Me, \Bi, \JB.
  Estimating the nature and the horizontal and vertical positions of 3D
  magnetic sources using Euler deconvolution,
  \emph{Geophysics},
  \DOI{10.1190/geo2012-0515.1}.
  & ~
  \\
  ~ &
  \Bi, \Val, \Me.
  Polynomial equivalent layer,
  \emph{Geophysics},
  \DOI{10.1190/geo2012-0196.1}.
  & ~
  \\
\Year{2012}  &
  \Me, \Val.
  Robust 3D gravity gradient inversion by planting anomalous densities,
  \emph{Geophysics},
  \DOI{10.1190/geo2011-0388.1}.
  &
  \GitHub{pinga-lab/paper-planting-densities}
\end{EntriesTableExtra}


\subsection{Peer-reviewed Conference Proceedings}

\begin{EntriesTableExtra}
\Year{2014}  &
  \Figura, \Val, \Me, \Bi, \JB.
  A Single Euler Solution Per Anomaly,
  \emph{76th EAGE Conference and Exhibition 2014},
  \DOI{10.3997/2214-4609.20140891}.
  & ~
  \\
\Year{2013}  &
  \Me, \Bi, \Val.
  Modeling the Earth with Fatiando a Terra,
  \emph{Proceedings of the 12th Python in Science Conference}.
  \DOI{10.25080/Majora-8b375195-010}.
  &
  \OA
  \GitHub{leouieda/scipy2013}
  \Slides{www.leouieda.com/scipy2013/?theme=night}
  \Youtube{Ec38h1oB8cc}
  \\
\Year{2012}  &
  \Me, \Val.
  Use of the ``shape-of-anomaly'' data misfit in 3D inversion by planting
  anomalous densities,
  \emph{SEG Technical Program Expanded Abstracts},
  \DOI{10.1190/segam2012-0383.1}.
  &
  \GitHub{leouieda/seg2012}
  \SlidesDOI{10.6084/m9.figshare.156864}
  \\
  ~ &
  \Dio, \Me, \YLi, \Val, \BragaVale, \Angeli, \Peres.
  Iron ore interpretation using gravity-gradient inversions in the Carajás, Brazil.
  \emph{SEG Technical Program Expanded Abstracts},
  \DOI{10.1190/segam2012-0525.1}.
  &
  \SlidesDOI{10.6084/m9.figshare.156865}
  \\
\Year{2011}  &
  \Me, \Everton, \Carla, \Eder.
  Optimal forward calculation method of the Marussi tensor due to a geologic
  structure at GOCE height,
  \emph{Proceedings of the 4th International GOCE User Workshop}.
  &
  \GitHub{leouieda/goce2011}
  \PosterDOI{10.6084/m9.figshare.92624}
  \\
  ~ &
  \Me, \Val.
  Robust 3D gravity gradient inversion by planting anomalous densities,
  \emph{SEG Technical Program Expanded Abstracts},
  \DOI{10.1190/1.3628201}.
  &
  \GitHub{leouieda/seg2011}
  \SlidesDOI{10.6084/m9.figshare.156863}
  \\
  ~ &
  \Me, \Val.
  3D gravity inversion by planting anomalous densities.
  \emph{12th International Congress of the Brazilian Geophysical Society},
  \DOI{10.1190/sbgf2011-179}.
  &
  \GitHub{leouieda/sbgf2011}
  \SlidesDOI{10.6084/m9.figshare.156861}
  \\
  ~ &
  \Me, \Val.
  3D gravity gradient inversion by planting density anomalies.
  \emph{73th EAGE Conference and Exhibition incorporating SPE EUROPEC},
  \DOI{10.3997/2214-4609.20149567}.
  &
  \GitHub{leouieda/eage2011}
  \PosterDOI{10.6084/m9.figshare.92624}
  \\
  ~ &
  \Dio, \Me, \Val, \BragaVale, \Gomes.
  In-depth imaging of an iron orebody from Quadrilatero Ferrifero using 3D
  gravity gradient inversion,
  \emph{SEG Technical Program Expanded Abstracts},
  \DOI{10.1190/1.3628219}.
  & ~
  \\
  ~ &
  \Dio, \Val, \Me, \BragaVale.
  Inversão de Dados de Aerogradiometria Gravimétrica 3D-FTG Aplicada a
  Exploração Mineral na Região do Quadrilátero Ferrífero,
  \emph{12th International Congress of the Brazilian Geophysical Society},
  \DOI{10.1190/sbgf2011-243}.
  & ~
\end{EntriesTableExtra}

\subsection{Non-peer-reviewed Papers}

\begin{EntriesTableExtra}
\Year{2017}  &
  \Me.
  Step-by-step NMO correction,
  \emph{The Leading Edge},
  \DOI{10.1190/tle36020179.1}.
  &
  \OA
  \GitHub{pinga-lab/nmo-tutorial}
  \\
\Year{2014}  &
  \Me, \Bi, \Val.
  Geophysical tutorial: Euler deconvolution of potential-field data,
  \emph{The Leading Edge},
  \DOI{10.1190/tle33040448.1}.
  &
  \OA
  \GitHub{pinga-lab/paper-tle-euler-tutorial}
\end{EntriesTableExtra}

\subsection{Preprints}

\begin{EntriesTableExtra}
\Year{2019}  &
  \LBarba, \JBazan, \JBrown, \RGuimera, \MGymrek, \AHanna, \Lindsey, \KHuff, \DKatz,
  \CMadan, \KMoerman, \KNiemeyer, \JPoulson, \PPrins, \KRam, \ARokem, \Arfon,
  \GThiruvathukal, \KThyng, \Me, \BWilson, \Yehudi.
  Giving software its due through community-driven review and publication.
  \emph{OSF Preprints}.
  \DOI{10.31219/osf.io/f4vx6}
  &
  \OA
\end{EntriesTableExtra}

%%%%%%%%%%%%%%%%%%%%%%%%%%%%%%%%%%%%%%%%%%%%%%%%%%%%%%%%%%%%%%%%%%%%%%%%%%%%%%%
\section{Presentations}

\subsection{Invited \& Keynotes}

\begin{EntriesTableExtra}
\Year{2022}  &
  \Me.
  Getting started with Open Science,
  \emph{SPIN SPIN-ITN: Seismological Parameters and Instrumentation},
  Online.
  &
  \GitHub{leouieda/2022-05-06-spin-open-science}
  \Slides{www.leouieda.com/2022-05-06-spin-open-science}
  \\
\Year{2021}  &
  \Me, \LLi, \Santiago, \Agustina.
  Design useful tools that do one thing well and work together: rediscovering
  the UNIX philosophy while building the Fatiando a Terra project,
  \emph{AGU 2021},
  Online.
  &
  \GitHub{fatiando/agu2021}
  \Slides{www.fatiando.org/agu2021}
  \\
  ~ &
  \Me, \Santiago.
  Python-based workflows for small-to-medium sized data: what works, what
  doesn't, and what can be improved,
  \emph{AGU 2021},
  Online.
  &
  \GitHub{compgeolab/agu2021}
  \Slides{www.compgeolab.org/agu2021}
  \\
  ~ &
  \Me.
  Academia e software livre: Desafios e oportunidades no Brasil e no exterior,
  \emph{National Observatory's SEG and EAGE Student Chapter},
  Rio de Janeiro, Brazil.
  &
  \GitHub{leouieda/2021-07-22-on}
  \Slides{www.leouieda.com/2021-07-22-on}
  \Youtube{r2x-DN6laj8}
  \\
  ~ &
  \Me, \Santiago, \Agustina.
  Open-science for gravimetry: tools, challenges, and opportunities,
  \emph{GFZ Helmholtz Centre Potsdam},
  Germany.
  &
  \GitHub{leouieda/2021-06-22-gfz}
  \SlidesDOI{10.6084/m9.figshare.14838477}
  \Youtube{z-5dvWfB\_SM}
  \\
  ~ &
  \Me, \Santiago, \Agustina.
  Fatiando a Terra: Open-source tools for geophysics,
  \emph{Geophysical Society of Houston},
  Houston, USA.
  &
  \GitHub{fatiando/2021-gsh}
  \Slides{www.fatiando.org/2021-gsh}
  \\
\Year{2020}  &
  \Me.
  Geophysical research powered by open-source,
  \emph{Christian Albrechts Universität zu Kiel},
  Kiel, Germany.
  &
  \GitHub{leouieda/2020-07-01-kiel}
  \Slides{www.leouieda.com/2020-07-01-kiel}
  \\
  ~ &
  \Me.
  Geophysical research powered by open-source,
  \emph{Departamento de Geofísica, IAG, Universidade de São Paulo},
  São Paulo, Brazil.
  &
  \GitHub{leouieda/2020-06-18-usp}
  \Slides{www.leouieda.com/2020-06-18-usp}
  \Youtube{VqI8BX1Yg54}
  \\
  ~ &
  \Me.
  Geophysical research powered by open-source,
  \emph{Technische Universität Bergakademie Freiberg},
  Freiberg, Germany.
  &
  \GitHub{leouieda/2020-06-04-freiberg}
  \Slides{www.leouieda.com/2020-06-04-freiberg}
  \\
  ~ &
  \Me.
  Geophysical research powered by open-source,
  \emph{Geographic Data Science Lab, University of Liverpool},
  Liverpool, UK.
  &
  \GitHub{leouieda/liverpool-gdsl-2020}
  \Slides{www.leouieda.com/liverpool-gdsl-2020}
  \\
\Year{2017}  &
  \Me, \Paul.
  Nurturing reliable and robust open-source scientific software,
  \emph{AGU Fall Meeting 2017},
  New Orleans, USA.
  &
  \Youtube{0GO4ZZ5Ry6M}
  \\
\Year{2016}  &
  \Me.
  Fatiando a Terra: construindo uma base para ensino e pesquisa de geofísica,
  \emph{Observatório Nacional},
  Rio de Janeiro, Brazil.
  &
  \SlidesDOI{10.6084/m9.figshare.1381870}
  \\
\Year{2015}  &
  \Me.
  Fatiando a Terra: construindo uma base para ensino e pesquisa de geofísica,
  \emph{Universidade de São Paulo},
  São Paulo, Brazil.
  &
  \SlidesDOI{10.6084/m9.figshare.1381870}
  \\
\end{EntriesTableExtra}

\subsection{Other Presentations}

\begin{EntriesTableExtra}
\Year{2021}  &
  \Me, \Santiago, \Agustina, \LPerozzi, \MWieczorek.
  Harmonica and Boule: Modern Python tools for geophysical gravimetry,
  \emph{EGU 2021},
  Online.
  \DOI{10.5194/egusphere-egu21-8291}.
  &
  \GitHub{fatiando/egu2021}
  \\
\Year{2020}  &
  \Me, \Santiago.
  Evaluating the accuracy of equivalent-source predictions using
  cross-validation,
  \emph{EGU 2020},
  Vienna, Austria.
  \DOI{10.5194/egusphere-egu2020-15729}.
  &
  \SlidesDOI{10.6084/m9.figshare.12245372}
  \\
\Year{2019}  &
  \Me, \Paul.
  PyGMT: Accessing the Generic Mapping Tools from Python,
  \emph{AGU 2019},
  San Francisco, USA.
  &
  \PosterDOI{10.6084/m9.figshare.11320280}
  \\
  ~ &
  \Me.
  Building the foundations for open-source geophysics,
  \emph{\LIVEARTH, \LIV},
  UK.
  &
  \SlidesDOI{10.6084/m9.figshare.10255832}
  \\
\Year{2018}  &
  \Me, \Eric, \Paul, \David.
  Coupled Interpolation of Three-component GPS Velocities,
  \emph{AGU 2018},
  Washington DC, USA.
  &
  \PosterDOI{10.6084/m9.figshare.7440683}
  \\
  ~ &
  \Me.
  Machine Learning Lessons for Geophysics,
  \emph{Department of Earth Sciences, \UHM},
  Honolulu, USA.
  &
  \SlidesDOI{10.6084/m9.figshare.7203344}
  \\
  ~ &
  \Me, \Paul.
  Building an object-oriented Python interface for the Generic Mapping Tools,
  \emph{Scipy 2018},
  Austin, USA.
  &
  \SlidesDOI{10.6084/m9.figshare.6814052}
  \Youtube{6wMtfZXfTRM}
  \\
  ~ &
  \Me, \David, \Paul.
  Joint Interpolation of 3-component GPS Velocities Constrained by
  Elasticity,
  \emph{AOGS $15^{th}$ Annual Meeting},
  Honolulu, USA.
  &
  \SlidesDOI{10.6084/m9.figshare.6387467}
  \\
  ~ &
  \Me, \Paul.
  Integrating the Generic Mapping Tools with the Scientific Python Ecosystem,
  \emph{AOGS $15^{th}$ Annual Meeting},
  Honolulu, USA.
  &
  \PosterDOI{10.6084/m9.figshare.6399944}
  \\
\Year{2017}  &
  \Me, \Paul.
  A modern Python interface for the Generic Mapping Tools,
  \emph{AGU Fall Meeting 2017},
  New Orleans, USA.
  &
  \PosterDOI{10.6084/m9.figshare.5662411}
  \\
  ~  &
  \Me, \Paul.
  Bringing the Generic Mapping Tools to Python,
  \emph{Scipy 2017},
  Austin, USA.
  &
  \SlidesDOI{10.6084/m9.figshare.7635833}
  \Youtube{93M4How7R24}
  \\
  ~ &
  \Me.
  Inverting gravity to map the Moho: A new method and the open source
  software that made it possible,
  \emph{Department of Geology and Geophysics, \UHM},
  Honolulu, USA.
  &
  \SlidesDOI{10.6084/m9.figshare.4779766}
  \\
\Year{2014}  &
  \Me, \Bi, \Val.
  Using Fatiando a Terra to solve inverse problems in geophysics,
  \emph{Scipy 2014},
  Austin, USA.
  &
  \PosterDOI{10.6084/m9.figshare.1089987}
  \\
  ~ &
  \Me, \Val.
  Gravity inversion in spherical coordinates using tesseroids,
  \emph{EGU General Assembly 2014},
  Vienna, Austria.
  &
  \SlidesDOI{10.6084/m9.figshare.1155457}
  \\
\Year{2013}  &
  \Me, \Bi, \Val.
  Modeling the Earth with Fatiando a Terra,
  \emph{Scipy 2013},
  Austin, USA.
  \DOI{10.25080/Majora-8b375195-010}.
  &
  \Slides{www.leouieda.com/scipy2013/?theme=night}
  \Youtube{Ec38h1oB8cc}
  \\
  ~ &
  \Me, \Val.
  3D magnetic inversion by planting anomalous densities,
  \emph{AGU Meeting of the Americas},
  Cancun, Mexico.
  &
  \SlidesDOI{10.6084/m9.figshare.703651}
  \\
\Year{2012}  &
  \Dio, \Me, \YLi, \Val, \BragaVale, \Angeli, \Peres.
  Iron ore interpretation using gravity-gradient inversions in the Carajás,
  Brazil,
  \emph{SEG Annual Meeting 2012},
  Las Vegas, USA.
  \DOI{10.1190/segam2012-0525.1}.
  &
  \SlidesDOI{10.6084/m9.figshare.156865}
  \\
  ~ &
  \Me, \Val.
  Use of the ``shape-of-anomaly'' data misfit in 3D inversion by planting
  anomalous densities,
  \emph{SEG Annual Meeting 2012},
  Las Vegas, USA.
  \DOI{10.1190/segam2012-0383.1}.
  &
  \SlidesDOI{10.6084/m9.figshare.156864}
  \\
  ~ &
  \Me, \Val.
  Rapid 3D inversion of gravity and gravity gradient data to test geologic
  hypotheses,
  \emph{International Symposium on Gravity, Geoid and Height Systems},
  Venice, Italy.
  &
  \SlidesDOI{10.6084/m9.figshare.156859}
  \\
\Year{2011}  &
  \Me, \Val.
  Robust 3D gravity gradient inversion by planting anomalous densities,
  \emph{SEG Annual Meeting 2011},
  San Antonio, USA.
  \DOI{10.1190/1.3628201}.
  &
  \SlidesDOI{10.6084/m9.figshare.156863}
  \\
  ~ &
  \Me, \Val.
  3D gravity inversion by planting anomalous densities,
  \emph{Internation Congress of the Brazilian Geophysical Society},
  Rio de Janeiro, Brazil.
  \DOI{10.1190/sbgf2011-179}.
  &
  \SlidesDOI{10.6084/m9.figshare.156861}
  \\
  ~ &
  \Me, \Everton, \Carla, \Eder.
  Optimal forward calculation method of the Marussi tensor due to a geologic
  structure at GOCE height,
  \emph{4th International GOCE User Workshop},
  Munich, Germany.
  &
  \PosterDOI{10.6084/m9.figshare.92624}
  \\
  ~ &
  \Me, \Val.
  3D gravity gradient inversion by planting density anomalies,
  \emph{73th EAGE Conference and Exhibition incorporating SPE EUROPEC},
  Vienna, Austria.
  \DOI{10.3997/2214-4609.20149567}.
  &
  \PosterDOI{10.6084/m9.figshare.91511}
  \\
\Year{2010}  &
  \Me, \Naomi, \Carla.
  Computation of the gravity gradient tensor due to topographic masses using
  tesseroids,
  \emph{AGU Meeting of the Americas},
  Foz do Iguaçu, Brazil.
  &
  \SlidesDOI{10.6084/m9.figshare.156858}
  \\
\Year{2008}  &
  \Me, \Naomi.
  Utilização de tesseróides na modelagem de dados de gradiometria
  gravimétrica,
  \emph{XIII Simpósio de Iniciação Científica do IAG-USP},
  São Paulo, Brazil.
  &
  \PosterDOI{10.6084/m9.figshare.4779760}
  \\
\Year{2006}  &
  \Me, \Manoel.
  Paleomagnetismo e mineralogia magnética dos diques cambrianos de Maravilhas
  e Prata (PB),
  \emph{XI Simpósio de Iniciação Científica do IAG/USP},
  São Paulo, Brazil.
  &
  \PosterDOI{10.6084/m9.figshare.4779769}
\end{EntriesTableExtra}

%%%%%%%%%%%%%%%%%%%%%%%%%%%%%%%%%%%%%%%%%%%%%%%%%%%%%%%%%%%%%%%%%%%%%%%%%%%%%%%
\section{Miscellaneous}

\subsection{Professional society membership}

\begin{EntriesTable}
  \Duration{2022}{\Ongoing} & Society of Research Software Engineering
  \\
  \Duration{2020}{\Ongoing} & Royal Astronomical Society
  \\
  \Duration{2014}{\Ongoing} & Software Underground
  \\
  \Duration{2014}{\Ongoing} & European Geosciences Union
  \\
  \Duration{2010}{\Ongoing} & American Geophysical Union
  \\
  \Duration{2011}{2019} & Society of Exploration Geophysicists
\end{EntriesTable}

\subsection{Languages}

\TablePad
\begin{tabularx}{\textwidth}{@{}p{0.15\textwidth} p{0.85\textwidth}@{}}
  Portuguese & Native
  \\
  English & IELTS: CEFR Level C2 (mastery or proficiency) obtained in 2019
\end{tabularx}

%%%%%%%%%%%%%%%%%%%%%%%%%%%%%%%%%%%%%%%%%%%%%%%%%%%%%%%%%%%%%%%%%%%%%%%%%%%%%%%
\section{Glossary}

These are the meanings of the symbols used throughout this document:
\\
\TablePad
\begin{tabularx}{\textwidth}{@{}p{0.03\textwidth} p{0.97\textwidth}@{}}
  \aiOpenAccess & Indicates that a publication is open-access
  \\
  \faGithub & Link to a code repository on GitHub
  \\
  \faFilePdf & Link to an open-access PDF, usually a preprint or postprint
  \\
  \faYoutube & Link to a video on YouTube
  \\
  \faChartLine & Link to a data archive
  \\
  \faTv & Link to presentation slides
  \\
  \faImage & Link to a poster
\end{tabularx}

\end{document}
